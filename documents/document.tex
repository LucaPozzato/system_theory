\documentclass{article}

\usepackage{animate}

\usepackage{pdfpages}
\usepackage{graphicx}
\usepackage{subcaption}
\usepackage{amsmath}
\usepackage{amssymb}
\usepackage[colorlinks=true, linkcolor=black, urlcolor=blue]{hyperref}
\usepackage[italian]{babel}

\newcommand{\rom}[1]{\uppercase\expandafter{\romannumeral #1\relax}}

\begin{document}

\title{Prova integrativa di Teoria dei Sistemi - Dinamica Non Lineare}
\author{Luca Pozzato}
\date{27 gennaio 2025}

% add title and table of contents on the bottom of the title page
\maketitle
\vspace{15em}
\tableofcontents

\newpage

\section{Equazioni di stato}

\begin{figure}[ht]
    \centering
    \begin{subfigure}{0.50\textwidth}
        \includegraphics[width=\linewidth]{images/circuit_clean.png}
        \caption{Circuito senza annotazioni}
        \label{fig:circuito_clean}
    \end{subfigure}
    \hfill
    \begin{subfigure}{0.65\textwidth}
        \includegraphics[width=\linewidth]{images/circuit_annotated.png}
        \caption{Circuito annotato utilizzando le leggi di Kirchhoff}
        \label{fig:circuito_annotato}
    \end{subfigure}
    \caption{Circuito elettrico}
    \label{fig:side_by_side}
\end{figure}

Dato il circuito elettrico in Figura \ref{fig:side_by_side}, possiamo scrivere le equazioni di stato del sistema utilizzando le leggi di Kirchhoff e utilizzando le equazioni caratteristiche dell'induttore e del condensatore. Dato che:
\begin{align*}
     & x_1 = i_L                        \\
     & x_2 = v_C                        \\
     & i_R = x_1                        \\
     & i   = -\alpha x_2 + \beta x_2^3,
\end{align*}

Date anche le equazioni caratteristiche dell'induttore, del condensatore e del resistore:
\begin{align*}
     & v_L = L \frac{di_L}{dt} = L \dot{x_1}, \\
     & i_C = C \frac{dv_C}{dt} = C \dot{x_2}, \\
     & v_R = R i_R = R x_1.
\end{align*}

Dalla prima legge di Kirchhoff possiamo scrivere:
\begin{align*}
     & x_1 + i_C + i = 0,        \\
     & i_C           = -x_1 - i.
\end{align*}

Analogamente, dalla seconda legge:
\begin{align*}
     & x_2 - v_L - v_R = 0,           \\
     & v_L             = x_2 - v_R,   \\
     & v_L             = x_2 - R x_1.
\end{align*}

Infine, sostituendo le equazioni caratteristiche dell'induttore, otteniamo:

\begin{align*}
     & L \dot{x_1} = - R x_1 + x_2,                    \\
     & \dot{x_1}   = -\frac{R}{L} x_1 + \frac{x_2}{L}.
\end{align*}

Analogamente, per il condensatore:

\begin{align*}
     & C \dot{x_2} = -x_1 - i,                                                      \\
     & C \dot{x_2} = -x_1 - (-\alpha x_2 + \beta x_2^3),                            \\
     & C \dot{x_2} = -x_1 + \alpha x_2 - \beta x_2^3,                               \\
     & \dot{x_2}   = -\frac{x_1}{C} + \frac{\alpha}{C} x_2 - \frac{\beta}{C} x_2^3.
\end{align*}

Quindi, le equazioni di stato del sistema sono:

\begin{equation}
    \left\{
    \begin{aligned}
         & \dot{x_1} = -\frac{R}{L}x_1 + \frac{x_2}{L}                             \\
         & \dot{x_2} = -\frac{x_1}{C} + \frac{\alpha}{C}x_2 - \frac{\beta}{C}x_2^3
    \end{aligned}
    \right.
\end{equation}

\section{Oscillazione con R sufficientemente grande}

Analizzando la divergenza del sistema, si osserva che, per \(R\) sufficientemente grande, la divergenza \(\text{div}\,f = \frac{\partial f_1}{\partial x_1} + \frac{\partial f_2}{\partial x_2}\), dove \(f\) è il sistema (nel nostro caso \(f = [f_1, f_2] = [\dot{x_1}, \dot{x_2}]\)), non cambia segno. Infatti data la divergenza del sistema:

\begin{align*}
     & \text{div}(f) = \frac{d(-\frac{R}{L}x_1 + \frac{x_2}{L})}{dx_1} + \frac{d(-\frac{x_1}{C} + \frac{\alpha}{C}x_2 - \frac{\beta}{C}x_2^3)}{dx_2}, \\
     & \text{div}(f) = -\frac{R}{L} + \frac{\alpha}{C} - \frac{3\beta x_2^2}{C}.
\end{align*}

si può osservare che al variare di \(R\) si ha:
\begin{align*}
     & R \geq \frac{\alpha L}{C} \hspace{1em} \text{la divergenza non cambia segno, quindi } \nexists \text{ cicli} \\\\
     & R < \frac{\alpha L}{C} \hspace{1em} \text{la divergenza cambia segno, quindi } \exists \text{ cicli}
\end{align*}

Per \(R\) sufficientemente grande, la divergenza non cambia segno e quindi il circuito non può oscillare.
Inoltre analizzando il circuito da un punto di vista puramente logico, dato una resistenza \(R\) sufficientemente grande, il resistore si può approssimare ad un circuito aperto mostrato in figura \ref{fig:circuito_aperto}.

\begin{figure}[ht]
    \includegraphics[width=\linewidth]{images/circuit_r_big.png}
    \caption{Circuito aperto}
    \label{fig:circuito_aperto}
\end{figure}

Dato il circuito aperto, la corrente \(i_R\) risulta essere nulla e di conseguenza si verifica che \(i_L = x_1 = 0\). Dalle equazioni del sistema si può osservare che se \(x_1 = 0\), allora \(\dot{x_1} = 0\), ciò si verifica se e solo se \([x_1 = 0, x_2 = c]\), dove \(c\) è una costante, \([x_1 = 0, x_2 = c]\) è quindi l'unico equilibrio del circuito con resistenza \(R\) sufficientemente grande. Dato che il determinante della matrice Jacobiana è positivo, e la traccia è negativa, l'equilibrio risulta asintoticamente stabile; quindi per tempo \(t \to \infty\) il sistema tende a convergere all'equilibrio \([0, c]\). Durante il transitorio, risolvendo il sistema di equazioni differenziali, si ottiene la soluzione mostrata in figura \ref{fig:soluzione_ode}, e si verifica che non vi sono oscillazioni.

Per paragone in figura \ref{fig:oscillazione_circuito} è mostrato il circuito con resistenza \(R = -0.5\) e condizioni iniziali \(x_1 = 0.1\) e \(x_2 = 0.1\), data la presenza di un ciclo limite stabile, si può osservare che il sistema oscilla.

\begin{figure}[ht]
    \centering
    \begin{subfigure}{0.49\textwidth}
        \includegraphics[width=\linewidth]{images/system_r_infinite.png}
        \caption{Soluzione del sistema con resistenza \(R \to \infty\)}
        \label{fig:soluzione_ode}
    \end{subfigure}
    \hfill
    \begin{subfigure}{0.49\textwidth}
        \includegraphics[width=\linewidth]{images/system_periodicity.png}
        \caption{Soluzione del sistema con resistenza \(R = -0.5\)}
        \label{fig:oscillazione_circuito}
    \end{subfigure}
    \caption{Soluzioni del sistema}
    \label{fig:soluzioni}
\end{figure}

\section{Stati di equilibrio e biforcazioni}

Analizzando il sistema e ponendo l'equazioni del sistema a zero:
\begin{align*}
     & -\frac{R}{L}\overline{x_1} + \frac{\overline{x_2}}{L} = 0,                                        \\
     & -\frac{\overline{x_1}}{C} + \frac{\alpha}{C}\overline{x_2} - \frac{\beta}{C}\overline{x_2}^3 = 0.
\end{align*}

e risolvendo il sistema di equazioni, si ottengono i seguenti punti di equilibrio:

\begin{align*}
    (x_1, x_2)^\text{\rom{1}} & = (0, 0),                                                                                    \\
    (x_1, x_2)^\text{\rom{2}} & = \left(\sqrt{\frac{\alpha R - 1}{\beta R^3}}, \sqrt{\frac{\alpha R - 1}{\beta R}}\right),   \\
    (x_1, x_2)^\text{\rom{3}} & = \left(-\sqrt{\frac{\alpha R - 1}{\beta R^3}}, -\sqrt{\frac{\alpha R - 1}{\beta R}}\right).
\end{align*}

Utilizzando il metodo di linearizzazione, si può calcolare la matrice Jacobiana del sistema:
\[
    J(x_1, x_2) =
    \begin{bmatrix}
        -\frac{R}{L} & \frac{1}{L}                                \\\\
        -\frac{1}{C} & \frac{\alpha}{C} - \frac{3 \beta x_2^2}{C}
    \end{bmatrix}
\]

la traccia della matrice Jacobiana:
\[
    \text{tr}(J) = - \frac{R}{L} + \frac{\alpha}{C} - \frac{3 \beta x_2^2}{C}
\]

e il determinante della matrice Jacobiana:
\[
    \det(J) = \frac{3 R \beta x_2^2 - R \alpha + 1}{C L}
\]

sostituendo i punti di equilibrio nella matrice Jacobiana e calcolando la traccia e il determinante, si ottengono le seguenti condizioni per la stabilità:

Per il punto di equilibrio \(\text{\rom{1}} = (0, 0)\) si ottiene che la matrice Jacobiana risulta essere:

\[
    J^\text{\rom{1}} =
    \begin{bmatrix}
        -\frac{R}{L} & \frac{1}{L}      \\\\
        -\frac{1}{C} & \frac{\alpha}{C}
    \end{bmatrix}
\]

quindi la traccia e il determinante risultano essere:
\begin{align*}
     & \text{tr}(J^\text{\rom{1}}) = -\frac{R}{L} + \frac{\alpha}{C}, \\
     & \det(J^\text{\rom{1}})      = \frac{1 - \alpha R}{C L}.
\end{align*}

da cui si può osservare che la traccia risulta negativa per \(R > \frac{\alpha L}{C}\) e il determinante positivo per \(R < \frac{1}{\alpha}\), quindi si può definire la seguente tabella:

\begin{table}[h!]
    \centering
    \begin{tabular}{|c|c|c|c|}
        \hline
        $ $                & $R < \frac{\alpha L}{C}$ & $\frac{\alpha L}{C} < R < \frac{1}{\alpha}$ & $\frac{1}{\alpha} < R$ \\ \hline
        $\text{tr}$        & $+$                      & $-$                                         & $-$                    \\ \hline
        $\det$             & $+$                      & $+$                                         & $-$                    \\ \hline
        $\text{stabilità}$ & $\text{instabile}$       & $\text{asintoticamente stabile}$            & $\text{sella}$         \\ \hline
    \end{tabular}
    \caption{Stabilità del punto di equilibrio \(\text{\rom{1}}\)}
\end{table}

Da questa tabella si può anche notare che esiste una biforcazione di Hopf supercritica per \(R = \frac{\alpha L}{C}\) dato che la traccia cambia segno e il determinante rimane positivo, si intuisce che la biforcazione sia supercritica dato che il punto di equilibrio \(\text{\rom{1}}\) diventa asintoticamente stabile e per \(R > \frac{\alpha L}{C}\) non ci possono essere cicli limite.

Dato che il punto di equilibrio \rom{2} e \rom{3} sono simmetrici rispetto all'origine e la matrice Jacobiana è equivalante per entrambi i punti, si ottiene che:

\[
    J^\text{\rom{2}} = J^\text{\rom{3}} =
    \begin{bmatrix}
        -\frac{R}{L} & \frac{1}{L}               \\\\
        -\frac{1}{C} & \frac{3 - 2 \alpha R}{CR}
    \end{bmatrix}
\]

da cui si ottiene che la traccia e il determinante risultano essere:
\begin{align*}
     & \text{tr}(J^\text{\rom{2}}) = -\frac{R}{L} + \frac{3 - 2 \alpha R}{C R}, \\\\
     & \det(J^\text{\rom{2}})      = \frac{2(\alpha R - 1)}{CL}.
\end{align*}

da cui si può osservare che il determinante risulta essere positivo per \(R > \frac{1}{\alpha}\) e la traccia negativa per:

\begin{align*}
     & \sqrt{\frac{L(\alpha^2L + 3C)}{C^2}} - \frac{\alpha L}{C} < R,       \\
     & - \sqrt{\frac{L(\alpha^2L + 3C)}{C^2}} - \frac{\alpha L}{C} < R < 0.
\end{align*}

per leggibilità definisco \(R^\text{\rom{1}} = - \sqrt{\frac{L(\alpha^2L + 3C)}{C^2}} - \frac{\alpha L}{C}\) e \(R^\text{\rom{2}} = \sqrt{\frac{L(\alpha^2L + 3C)}{C^2}} - \frac{\alpha L}{C}\)

inoltre gli equilibri esistono solo per alcuni valori di \(R\), si ha infatti che:
\begin{align*}
     & \frac{\alpha R - 1}{\beta R^3} > 0 \hspace{1em} \text{e} \hspace{1em} \frac{\alpha R - 1}{\beta R} > 0, \\
     & \text{si ha che} \hspace{1em} R > \frac{1}{\alpha} \hspace{1em} \text{e} \hspace{1em} R < 0.
\end{align*}

quindi si può definire la seguente tabella:

\begin{table}[h!]
    \centering
    \begin{tabular}{|c|c|c|c|c|c|}
        \hline
        $ $                & $R < R^\text{\rom{1}}$ & $R^\text{\rom{1}} < R < 0$ & $0 < R < R^\text{\rom{2}}$  & $R^\text{\rom{2}} < R < \frac{1}{\alpha}$ & $\frac{1}{\alpha} < R$           \\ \hline
        $\text{tr}$        & $+$                    & $-$                        & $+$                         & $-$                                       & $-$                              \\ \hline
        $\det$             & $-$                    & $-$                        & $-$                         & $-$                                       & $+$                              \\ \hline
        $\text{stabilità}$ & $\text{sella}$         & $\text{sella}$             & $\nexists\text{ equilibri}$ & $\nexists\text{ equilibri}$               & $\text{asintoticamente stabile}$ \\ \hline
    \end{tabular}
    \caption{Stabilità del punto di equilibrio \(\text{\rom{2}}\) e \(\text{\rom{3}}\)}
\end{table}

Dall'analisi di questi punti di equilibrio si può osservare che oltre alla biforcazione di Hopf, esiste una biforcazione a forcone supercritica dato che il punto di equilibrio \(\text{\rom{1}}\) per \(R > \frac{1}{\alpha}\) diventa una sella, invece i due punti di equilibrio \(\text{\rom{2}}\) e \(\text{\rom{3}}\) che non esistono per \(R < \frac{1}{\alpha}\) diventano asintoticamente stabili per \(R > \frac{1}{\alpha}\).

\section{Invarianti del sistema al variare di R}

In figura \ref{fig:invarianti} e in \ref{fig:sistema_grafo} sono mostrati gli invarianti (equilibri e cicli) del sistema nello spazio \(R, x_1, x_2\), in verde gli attrattori, in rosso i repulsori e in blu le selle. Si può osservare che per \(R \geq \frac{\alpha L}{C}\) non vi sono cicli limite, mentre per \(R < \frac{\alpha L}{C} = 0.04 \), si ha un ciclo limite stabile che deriva dalla biforcazione di Hopf supercritica. Con \(R = \frac{1}{\alpha} = 25\) si può notare la biforcazione a forcone supercritica, dove il punto di equilibrio \(\text{\rom{1}}\) diventa una sella e i punti di equilibrio \(\text{\rom{2}}\) e \(\text{\rom{3}}\) diventano asintoticamente stabili.

\begin{figure}
    \centering
    \includegraphics[width=\linewidth]{images/invarianti_1.png}
    \caption{Invarianti del sistema nello spazio (\(R, x_1, x_2\))}
    \label{fig:invarianti}
\end{figure}

\begin{figure}
    \centering
    \begin{subfigure}{0.45\textwidth}
        \includegraphics[width=\linewidth]{images/system_x2.png}
        \caption{Invarianti del sistema nello spazio (\(R, x_2\))}
        \label{fig:sistema_x2}
    \end{subfigure}
    \hfill
    \begin{subfigure}{0.45\textwidth}
        \includegraphics[width=\linewidth]{images/system_x1.png}
        \caption{Invarianti del sistema nello spazio (\(R, x_1\))}
        \label{fig:sistema_x1}
    \end{subfigure}
    \caption{Invarianti del sistema}
    \label{fig:sistema_grafo}
\end{figure}

\newpage

\section{Numero biforcazioni}

Dall'analisi del sistema si può anche notare che oltre alla biforcazione di Hopf supercritica e alla biforcazione a forcone supercritica, vi sono anche 2 biforcazioni omocline, si può notare nella figura \ref{fig:invarianti} che le due selle simmetriche, associate agli equilibri \rom{2} e \rom{3}, si avvicinano progressivamente fino a intersecarsi con il ciclo limite stabile, trasformandolo in un'orbita omoclinica.

\begin{table}[h!]
    \centering
    \begin{tabular}{|c|c|}
        \hline
        $\text{forcone}$           & $1$ \\ \hline
        $\text{Hopf}$              & $1$ \\ \hline
        $\text{omoclina}$          & $2$ \\ \hline
        $\text{tangente di cicli}$ & $0$ \\ \hline
    \end{tabular}
    \caption{Tabella delle biforcazioni}
\end{table}

\section{Caos, sezione di Poincaré ed esponenti di Lyapunov}

Dato che la resistenza R viene qui definita in funzione del tempo come: \\ \(R(t) = \overline{R}(1 + 0.25sin(\omega t))\), analizzando il sistema con simulazioni numeriche utilizzando il risolutore ODE15s di MATLAB, si può osservare come per \(\omega = 0.02\), \(\overline{R} = -0.64\) e condizioni iniziali pari a \([x_1 = 1, x_2 = 1]\) il sistema risulta essere caotico. In figura \ref{fig:chaos} è mostrato l'attrattore caotico nello spazio (\(x_1, x_2, R\)) e la sezione di Poincaré per \(R = \overline{R}\)

\begin{figure}[ht]
    \centering
    \includegraphics[width=\linewidth]{images/chaos.png}
    \caption{Attrattore caotico nello spazio (\(x_1, x_2, R\)) e sezione di Poincaré}
    \label{fig:chaos}
\end{figure}

Sempre utilizzando MATLAB è stato possibile anche simulare il sistema e creare un animazione del sistema nel tempo con due condizioni iniziali diverse, la prima con \([x_1 = 1, x_2 = 1]\) (in rosso) e la seconda con \([x_1 = 1.000001, x_2 = 1]\) (in blu), si può osservare come le due traiettorie divergono nel tempo, confermando che il sistema è sensibile alle condizioni iniziali, inoltre mostrando anche fenomeni di stretching e folding. Utilizzando Adobe Acrobat Reader per leggere questo documento, è possibile visualizzare l'animazione. Utilizzando inoltre l'applicazione \href{https://it.mathworks.com/matlabcentral/fileexchange/4628-calculation-lyapunov-exponents-for-ode}{Calculation Lyapunov Exponents for ODE} su MATLAB è stato possibile calcolare gli esponenti di Lyapunov del sistema, ottenendo i seguenti valori: \(\lambda_1 = 0.003050\), \(\lambda_2 = -0.116506\), confermando che il sistema è caotico. In figura \ref{fig:lyapunov} sono mostrati gli esponenti di Lyapunov nel tempo e in figura \ref{fig:animazione} l'animazione del sistema nel tempo.

\begin{figure}[ht]
    \centering
    \includegraphics[width=1\linewidth]{images/lyapunov.png}
    \caption{Esponenti di Lyapunov}
    \label{fig:lyapunov}
\end{figure}

% \vspace{2em}

\begin{figure}[ht]
    \centering
    \animategraphics[loop,autoplay,width=\linewidth]{15}{images/animation/frame_}{763}{763}
    \caption{Animazione del sistema nel tempo}
    \label{fig:animazione}
\end{figure}

\newpage

\section{Diagramma di biforcazione}

Infine, si può osservare in figura \ref{fig:diagramma_biforcazione_singolo} il diagramma della stabilità dei singoli equilibri e il numero degli equilibri nello spazio (\(R, \alpha\)) e in figura \ref{fig:diagramma_biforcazione} il diagramma di biforcazione nello spazio (\(R, \alpha\))

\begin{figure}
    \centering
    \includegraphics[width=0.95\linewidth]{images/bifurcation_single.png}
    \caption{Diagramma della stabilità dei singoli equilibri nello spazio (\(R,\alpha\))}
    \label{fig:diagramma_biforcazione_singolo}
\end{figure}

\begin{figure}
    \centering
    \includegraphics[width=0.95\linewidth]{images/bifurcation_diagram.png}
    \caption{Diagramma di biforcazione nello spazio (\(R, \alpha\))}
    \label{fig:diagramma_biforcazione}
\end{figure}

\newpage
Per ogni regione viene infine riportato in figura \ref{fig:sketch_qualitativo} uno sketch qualitativo del comportamento del sistema. Gli sketch e l'analisi degli equilibri sono stati realizzati utilizzando l'applicazione {\href{https://it.mathworks.com/matlabcentral/fileexchange/91705-phase-plane-and-slope-field-apps}{Phase Plane and Slope Field}} su MATLAB.

\begin{figure}[ht]
    \centering
    \begin{minipage}{0.24\linewidth}
        \centering
        \includegraphics[width=\textwidth]{images/system_1.png}
        \subcaption{Regione 1,\\\(R = -2, \alpha = 2\)\\2 selle instabili\\1 spirale instabile}
    \end{minipage}
    \begin{minipage}{0.24\linewidth}
        \centering
        \includegraphics[width=\textwidth]{images/system_2.png}
        \subcaption{Regione 2,\\\(R = 0.5, \alpha = 1\)\\1 spirale instabile}
    \end{minipage}
    \begin{minipage}{0.24\linewidth}
        \centering
        \includegraphics[width=\textwidth]{images/system_3.png}
        \subcaption{Regione 3,\\\(R = 0.531, \alpha = 2.09419\)\\2 nodi instabili\\1 sella instabile}
    \end{minipage}
    \begin{minipage}{0.24\linewidth}
        \centering
        \includegraphics[width=\textwidth]{images/system_6.png}
        \subcaption{Regione 6,\\\(R = -3, \alpha = -3\)\\1 sella instabile}
    \end{minipage}

    \vspace{1em}

    \begin{minipage}{0.24\linewidth}
        \centering
        \includegraphics[width=\textwidth]{images/system_4.png}
        \subcaption{Regione 4,\\\(R = -0.531062, \alpha = -0.531062\)\\2 selle instabili\\1 centro}
    \end{minipage}
    \begin{minipage}{0.24\linewidth}
        \centering
        \includegraphics[width=\textwidth]{images/system_5.png}
        \subcaption{Regione 5,\\\(R = 3, \alpha = 3\)\\2 nodi stabili\\1 sella instabile}
    \end{minipage}
    \begin{minipage}{0.24\linewidth}
        \centering
        \includegraphics[width=\textwidth]{images/system_7.png}
        \subcaption{Regione 7,\\\(R = -0.350701, \alpha = -1.03206\)\\2 selle instabili\\1 spirale stabile}
    \end{minipage}
    \begin{minipage}{0.24\linewidth}
        \centering
        \includegraphics[width=\textwidth]{images/system_8.png}
        \subcaption{Regione 8,\\\(R = 3, \alpha = -2\)\\1 spirale stabile}
    \end{minipage}

    \caption{Sketch qualitativo del comportamento del sistema}
    \label{fig:sketch_qualitativo}
\end{figure}

\end{document}